% Semantic Assistants - http://www.semanticsoftware.info/semantic-assistants
%
% This file is part of the Semantic Assistants architecture.
%
% Copyright (C) 2009, 2010, 2011, 2012, 2013 Semantic Software Lab, http://www.semanticsoftware.info
% The Semantic Assistants architecture is free software: you can
% redistribute and/or modify it under the terms of the GNU Affero General
% Public License as published by the Free Software Foundation, either
% version 3 of the License, or (at your option) any later version.
%   
% This program is distributed in the hope that it will be useful,
% but WITHOUT ANY WARRANTY; without even the implied warranty of
% MERCHANTABILITY or FITNESS FOR A PARTICULAR PURPOSE.  See the
% GNU Affero General Public License for more details.
% 
% You should have received a copy of the GNU Affero General Public License
% along with this program.  If not, see <http://www.gnu.org/licenses/>.


\chapter{Command-Line Client}
\label{sec:sacl:clc}
This is a simple example client to access the server from the command line.
It is located under \url{SemanticAssistants/Clients/CommandLine} and is meant to demonstrate and test to plug-in developers various Semantic Assistant functionalities.

\begin{enumerate}
\item To compile: \emph{ant compile}
\item To run: \emph{./runclient.sh}
\end{enumerate}

The \texttt{runclient.sh} script helps with the
class path setting, but also adds some difficulty with getting quotes right
when passing parameters to the program. For example, to list all
available services, you can run
\begin{verbatim}
    ./runclient.sh listall
\end{verbatim}
to query the server for all available NLP services. For the default
installation, you should see an output like:
\begin{verbatim}
    Retrieving service info from server...   done
    Listing services:
    Yahoo Search
    IR Information Extractor
    Person and Location Extractor
\end{verbatim}
Now you can invoke one of the services. For example, to extract all
person and location entities from a Wikipedia article, you can run
\begin{verbatim}
    ./runclient.sh invoke "\"Person and Location Extractor\"" \
    "docs=http://en.wikipedia.org/w/index.php?title=Christiane_Kubrick&printable=yes"
\end{verbatim}
If everything works, you will see the raw service response (in XML
format).  Note again that the server has to be running and both the
CSAL and command-line client must have been compiled successfully.

\section*{Connecting to any Server}
The user is able to specify the Server information (Host and Port) of
a local or distant server.  To achieve that the \url{params} part of the
command needs to be used.  The only extra info needed is appending the
following string to the end of the command:
\begin{verbatim}
    "params=(Host=<target Host>,Port=<target server port>)"
\end{verbatim}

For example:
\begin{verbatim}
    "params=(Host=localhost,Port=8080)"
\end{verbatim}

This parameter list may be added at every invocation.

\section*{Configuring Client Preferences}
The Semantic Assistant CSAL architecture makes it is possible to configure persistent server connection and runtime preferences for the command-line client via the \texttt{semassist-settings.xml} file described in section \ref{sec:pref_management}.
Run the following to see all configurations relevant to the command-line client. The output should be similar to this:
\begin{verbatim}
    ./runclient.sh listpref

    global preferences:
    lastCalledServer.port=8879
    lastCalledServer.host=minion.cs.concordia.ca
    server.port=8879
    server.host=minion.cs.concordia.ca
    server.port=8879
    server.host=assistant.cs.concordia.ca

    cmdline preferences:
\end{verbatim}

To then create new or override existing preferences in either the global or the client scopes, you can run something like the following:
\begin{verbatim}
    ./runclient.sh setpref cmdline server.host=localhost
    ./runclient.sh setpref cmdline server.port=8080
\end{verbatim}
Note that while any preference can be configured, only supported ones will take effect for the command-line client.
Only the following preferences are currently supported: \texttt{server.host}, \texttt{server.port}, \texttt{lastCalledServer.host} and \texttt{lastCalledServer.port}.


