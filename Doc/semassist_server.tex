\chapter{The \sa Server} 

\section{Starting the Server}
Type \texttt{ant run} in the \url{SemanticAssistants/Server} directory
to start the server. Please refer to Section~\ref{sec:inst-comp} for
more installation and compilation details. The server will
automatically load all available OWL service descriptions from the
default location \url{Resources/OwlServiceDescriptions} and publish
these to the clients.


\section{Testing using the Command Line Client}
To test if the server is running correctly and can be accessed from
the clients, we recommend you run some tests using the command-line
client described in Section~\ref{sec:sacl:clc}.


\section{Integrating New NLP Services}\label{sec:nlpservices}
For the server to know how to handle the different NLP services
offered through the architecture, it needs a \emph{description} of
each offered service. These are by default located in the
\url{SemanticAssistants/Resources/OwlServiceDescriptions}
directory. The GATE pipelines corresponding to these service
descriptions are located (by default) in
\url{Resources/GatePipelines}. The language service descriptions are
ontologies, building on the \emph{SemanticAssistants.owl} ontology,
which, in turn, extends the \emph{ConceptUpper.owl} ontology. Both of
these are located in \url{ont-repository} under
\url{SemanticAssistants/Resources}.

In order to create a new language service description, it is often
easier to copy an old one and edit
it. Prot\'{e}g\'{e}\footnote{Prot\'{e}g\'{e},
  \url{http://protege.stanford.edu}} is helpful as an ontology
editor. Most important is to define the parameters that can be passed
to this language service, as well as the description of the results
that should be passed back to the client.

In summary, to integrate a new NLP service, two steps are necessary:
\begin{enumerate}
\item Store the GATE pipeline implementing the service under
  \url{Resources/GatePipelines} (using GATE's \emph{Save Application State}
  or \emph{Export to Teamware} menu functions).
\item Develop an OWL service description for this pipeline.  For
  details on the OWL NLP description format, please refer to
  Section~\ref{sec:owl}.
\end{enumerate}