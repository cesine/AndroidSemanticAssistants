%% Semantic Assistants Documentation
%% 
%% This file is part of the Semantic Assistants architecture.
%%
%% Copyright (C) 2009, 2010, 2011 Semantic Software Lab, http://www.semanticsoftware.info
%%
%% The Semantic Assistants architecture is free software: you can
%% redistribute and/or modify it under the terms of the GNU Affero General
%% Public License as published by the Free Software Foundation, either
%% version 3 of the License, or (at your option) any later version.
%% 
%% This program is distributed in the hope that it will be useful,
%% but WITHOUT ANY WARRANTY; without even the implied warranty of
%% MERCHANTABILITY or FITNESS FOR A PARTICULAR PURPOSE.  See the
%% GNU Affero General Public License for more details.
%% 
%% You should have received a copy of the GNU Affero General Public License
%% along with this program.  If not, see <http://www.gnu.org/licenses/>.
%%

\chapter{Troubleshooting}
\section{Compiling}
\subsection{Eclipse Plug-In}
\paragraph{"Too many open files" Error.} This problem happens only on
Linux machines and is related to the default limitations of shell
resources in some system configurations. You get this error
when you're pushing various limits like open file descriptors to your
system e.g. compiling numerous classes in JVM:


\begin{lstlisting}[language=Ant,xleftmargin=8mm,columns=flexible,numbers=left]
compile:
    [javac] Compiling 26 source files to /home/user/Repository/semantic-assist/Clients/EclipsePlugin/bin
    [javac] error: error reading /home/user/Repository/semantic-assist/Clients/EclipsePlugin/src/info/semanticsoftware/semassist/client/eclipse/Activator.java; 
        /home/user/Repository/semantic-assist/Clients/EclipsePlugin/src/info/semanticsoftware/semassist/client/eclipse/Activator.java (Too many open files)
    ....
\end{lstlisting}

\noindent To fix this problem, first check your user limits with the
following command:
\begin{verbatim}
    ulimit -aH
\end{verbatim}
The \texttt{ulimit} is a Linux built-in shell command, which provides
control over the resources available to the shell and its child
processes. The result includes the ``open files'' limit imposed on
your system. This compilation error arises when your open files
size is less than what is required by the JVM to compile all the
classes, which is at least 6000. Thus, to bypass this problem, simply
increase the open files size to a size more than 6000 (note that this
typically requires administrator priviliges):
\begin{verbatim}
    sudo ulimit -n 8000
\end{verbatim}
To change the limit permanently, consult the documentation of your
distribution. E.g., under Ubuntu, you can increase the limit by
editing the file \verb=/etc/security/limits=:
\begin{verbatim}
    hard nofile 10000
    soft nofile 10000
\end{verbatim}

\section{Installation}
\subsection{Eclipse Plug-In}
\paragraph{The Plug-in does not show up in Eclipse.} This problem happens when the plug-in could not successfully install itself on your Eclipse. The most common cause of this is when you have not enough permissions on your Eclipse installation folder. For Eclipse plug-in to install itself, it needs write permission to the \texttt{configurations} folder. Login as root or obtain the adequate permissions to write into the Eclipse installation folder and restart application.
It is recommended to start Eclipse with \texttt{"-clean"} command line argument to clear any per-configuration cached data.